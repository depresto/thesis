\documentclass{article}
\usepackage{geometry}
\geometry{
  a4paper,
  total={170mm,257mm},
  left=20mm,
  top=20mm,
}
\usepackage{mathtools}
\usepackage{stackengine}
\usepackage{xeCJK}
\setCJKmainfont{Songti TC Light}
\usepackage{tikz}
\usetikzlibrary{calc, arrows}
\usepackage{graphicx}
\newcommand\hash{\mathbin{\mathpalette\xhash\relax}}
\newcommand{\xhash}[2]{\ooalign{%
  $#1\xxhash{#1}{-45}$\cr
  $#1\xxhash{#1}{45}$\cr
  }%
}
\newcommand{\xxhash}[2]{\rotatebox[origin=c]{#2}{$#1\parallel$}}

\begin{document}

\section*{共乘問題}

\begin{enumerate}
\def\labelenumi{\arabic{enumi}.}
  \item 可利用前述之model
  \item Rider為car pool中之共享車輛(M to M)
  \item 餐廳為共乘旅客之出發地點
  \item 客戶所在地(送貨點)為乘客之目的地
  \item 食物運送箱之客量為共乘車輛可搭載之人數
  \item 對乘客之QoS指標可含
    \begin{enumerate}
      \def\labelenumii{\arabic{enumii}.}
      \item Pick up time
      \item Travel time
      \item Arrival time
      \item 共乘人數
      \item 分攤費用
      \item 共乘路徑與最短路徑之差值等
    \end{enumerate}
\end{enumerate}

\subsection*{經多給定點之最短路徑問題(S to M)}
E.g. 機車接送學生 or 某員工開個pick up同仁一同上班,若為後者可進一步考慮公平性問題

\begin{enumerate}
\def\labelenumi{\arabic{enumi}.}
  \item 由A出發,目的地為Z
  \item 途中需經\(U_1, U_2, …, U_n\) (無需依序)之給定點 \((U_i, U_i') \in L’\)
  \item Modeled as \(G(N, L)\), 將\(U_1 ~ U_n\)利用node splitting技術產出artificial link集合\(L’\)
  \item 可將n必經點排序,再找出依序之最短路徑,但複雜度為\(n!\),過高
  \item
    \begin{align*}\tag{LP}
      min & \sum\limits_{p \in P_z} \sum\limits_{l \in L \cup L'} x_p s_{pl} a_l && \text{where } l = \text{const of link, } a_l = 0,\ \forall l \in L' \\
      & (P_z \text{: 所有由A至Z路徑所成之集合}) \\
      s.t. & \sum\limits_{p \in P_z} x_p = 1 \tag{1} \\
      & x_p = 0 \ or \ 1 && \forall p \in P_z \tag{2} \\
      \text{(To be relaxed)} & \sum\limits_{p \in P_z} x_p s_{pl} = 1 && \forall l \in L' \tag{3}
    \end{align*}
\end{enumerate}

\begin{itemize}
  \item 第(3)式保證最短路徑會經過 \(U_1,U_2,...U_n\)
  \item 經relaxed後,對應之arc weight (LR multipliers) 可為± or 0,若為負則LR中做短路徑問題即不可用Dijkstra,而須改用 e.g.~Bellman-Ford with max hop constraint (避免負迴圈)
\end{itemize}

\newpage
將前述經特定多點之最短路徑問題修改為每一\(U_i\)有一需經過之對應\(u_i\),同時所標路徑必須滿足先經\(U_i\)後經\(u_i\)

\begin{enumerate}
\def\labelenumi{\arabic{enumi}.}
  \item 一有趣之問題為合乎所有\(U_i\),均須前於\(u_i\)條件之permutations共有多少種?
  \item math formulation: (\(u_i\)亦經node splitting並增加artificial links\(\in L''\))
  
    \{令\(U_i\)之集合為\(V\), \(u_i\)之集合為\(U\),\(i \in \{1,2,...,n\}\)\}

    \begin{align*}
      min & \sum\limits_{p \in P_z} \sum\limits_{l \in L \cup L' \cup L''} x_p s_{pl} a_l \\
      s.t. & \sum\limits_{p \in P_z} x_p = 1 && for\ U_i \ \& \ u_i, i \in \{1,2,...,n\} \tag{1} \\
      & x_p = 0 \ or \ 1 && \forall p \in P_z \tag{2} \\
      & \sum\limits_{p \in P_z} x_p s_{pl} = y_l && \forall l \in L \cup L' \cup L'' \tag{3} \\
      & \text{(link \(l\) 若在最短路徑上,則 \(y_l = 1\),否則為 0)} \\
      & y_l = 1 && \forall l \in L' \cup L"" \tag{4} \\
      & \sum\limits_{p \in P_w} z_p s_{pl} \leq y_l && \forall w \in V \cup U = W \tag{5} \\
      & \text{(保證計算到 \(U_i\) 及 \(u_i\) 所用之路徑 \("z_p"\) 重疊於A至Z之路徑)} \\
      & \sum\limits_{p \in P_w} z_p = 1 && \forall w \in W \tag{6} \\
      & z_p = 0 \ or \ 1 \tag{7} \\
      & \sum\limits_{l \in L \cup L'} \sum\limits_{p \in Q_{U_i}} z_p s_{pl} a_l \leq \sum\limits_{l \in L \cup L'} \sum\limits_{p \in Q_{u_i}} z_p s_{pl} a_l && \forall i \in \{1,2,,...,n\} \tag{8} \\
      & \text{(保證\(U_i\)先\(u_i\)後)} \\
      & \text{(\(Q_{u_i}\)為A至\(U_i\)所有路徑之集合\(Q_{u_i},...,u_i,..., i \in \{1,2,3,...,n\} = R\))} \\
    \end{align*}
\end{enumerate}

於前面之 formulation 中可加入各客戶 \(u \in \{1,2,...,n\} = R\) 所要求之QoS,如此則需加入admission control機制

\begin{enumerate}
\def\labelenumi{\arabic{enumi}.}
  \item 定義客戶集 $\{1,2,...,n\} = R$
  \item 定義各客戶之“獨享”收費為$r_i,\ i \in R$
  \item 重新定義前頁中之$y_l,\ l \in L' \cup L''$為若$i \in R$被允入,則對應之2個$y_l$ (個別對應於$U_i$及$u_i$分裂後之 artificial links)之值設為1,否則為0
  \item 令 $e(U_i)$ 為 the link corresponding to that in L' \\
        令 $f(u_i)$ 為 the link corresponding to that in L''
  \item 目標函數改為 $max \sum\limits_{i \in R} y_{e(U_i)} r_i$
  \item Constraint (3) 改為 $\sum_{p \in P_z} x_p s_{pl} \leq y_l \hspace{10pt} \forall l \in L' \cup L''$ \\
        (若 i 被允入則獲 $r_i$ 收益,且A至Z之路徑必須經過 i 於 L' 及 L'' 對應之 links)
  \item Constraint (4) 改為 $y_{e(U_i)} = y_{f(u_i)} \hspace{23pt} \forall i \in R$ \\
        (i 於 L' 及 L'' 中對應 links 所對應之兩 $y_l$ 必須等值)
\end{enumerate}

\newpage
\subsection*{就前述共乘服務提出公平之付費機制}
(忽略以下論述,直接跳至下頁 $a \rightarrow a'$ 及 $b \rightarrow b' 之例$)

\begin{enumerate}
\def\labelenumi{\arabic{enumi}.}
  \item 可比較對應任一客戶 $i$ “獨享”與“共享”後之 QoS 差異,含前述五項之頭四項
  \item 總計費可考慮“單人依旅程計費”乘以 $\alpha_i (\alpha_i \leq 1)$ 以反映共乘服務帶給乘客之價值(當乘客人數始終為1時,$\alpha_i = 1$;當共享人數越多時,$\alpha_i$ 可當一convex且遞減曲線)
  \item 以2所計算出之總計費(先考慮 $\alpha_i = 1$ 之情況以簡化問題),再依 1 所計算出之 QoS 差異及
  \item 另可定義 $\beta_j$ 為 $\ j \in D$ 因共乘所”犧牲“之程度,則 $\alpha_i$ 可視為 $\overrightarrow{\beta_j},\ j \in D$ 之函數,當其他乘客之“犧牲”度高時,則 $\alpha_i$ 亦多愈高 $\Longrightarrow$
    \begin{enumerate}
    \def\labelenumi{\arabic{enumi}.}
      \item $\alpha_i$ 因 $\overrightarrow{B_j},\ i \in D$ 之維度升高而下降
      \item $\alpha_i$ 因 $\beta_j,\ j \in D$,之個別值提高而升高(其他乘客較大之犧牲應由提高之費用($\alpha_i$)予以補償)
    \end{enumerate}
    承4. $\beta_j$ 可擴充為 $\beta_{ji}$,即為“$i$ 對 $j$ 造成之不便“(or $j$ 對 $i$ 所付出之犧牲);$\alpha_i$ 為 $\overrightarrow{\beta_{ji}},\ j \in D$ 之函數且 $\alpha_i$ 因此造成提升之資費,應依 $\beta_{ji}$ 之大小而分配至相同 $j \in D$
\end{enumerate}

\begin{center}
  \begin{tikzpicture}[
    ->,>=stealth',
    shorten >=1pt,
    auto,
    node distance=3cm,
    thick,
    main node/.style={minimum height=2pt,circle,draw,font=\sffamily\Large\bfseries}]
    \node[main node] (0) {};
    \node[main node] (1) [below left of=0]  {};
    \node[main node] (2) [below right of=0] {};
    \node[main node] (3) [below of=1]       {};
    \node[main node] (4) [below of=2]       {};
    \node (0top) at ($(0) + (40:.7)$)   {$S$(出發地)};
    \node (1top) at ($(1) + (140:.5)$)  {$a$};
    \node (2top) at ($(2) + (50:.5)$)   {$b$};
    \node (3top) at ($(3) + (140:.5)$)  {$a'$};
    \node (4top) at ($(4) + (50:.5)$)   {$b'$};

    \draw[-] (0) to node [above left] {1} (1);
    \draw[-] (0) to node [above right] {1} (2);
    \draw[-] (1) to node [above] {1} (2);
    \draw[-] (1) to node [left] {5} (3);
    \draw[-] (2) to node [right] {5} (4);
    \draw[-] (1) to node [above left] {5} (4);
    \draw[-] (2) to node [above right] {5} (3);
    \draw[-] (3) to node [below] {1} (4);
  \end{tikzpicture}

  For $A$ to travel from $a$ to $a'$ \\
  \hspace{12pt} $B$ to travel from $b$ to $b'$
\end{center}

\begin{itemize}
  \item 若只有 $A$ 則費用為 $\overline{aa'} = 5$
  \item 若只有 $B$ 則費用為 $\overline{bb'} = 5$
  \item 最佳 car pool 路線為 \\
      Path 1: $S - b - a - b' - a'$,費用為 8,or \\
      Path 2: $S - b - a - a' - b'$,費用為 8 \\
      (若自接到第一個客戶起算則為 7) \\
      但前者 $A$ 之“費用”為 6,$B$ 為 6 \\
      而後者 $A$ 之“費用”為 5,$B$ 為 7 \\
      $\Longrightarrow$ 可考慮在min全程費用之同時亦令各客戶之“犧牲度”愈平均愈好
      \begin{itemize}
        \item 以上述$\overset{Path 1}{\text{路程分析}}$,$B$ 分攤之費用為 $\overset{\overline{ba}}{1} + \frac{1}{2} \times \overset{\overline{ab'}}{5}$,$A$ 則為 $\frac{1}{2} \times \overset{\overline{ab'}}{5} + \overset{\overline{ba'}}{1}$ \\
          $\Longrightarrow$ 各為 $3.5 < 5_{\hash}$
        \item 若為 Path 2,則 $A$ 分攤 2.5,$B$ 4.5
      \end{itemize}
\end{itemize}

\newpage
\subsection*{Final formulation}
\[
  \underset{p \in P}{min}\ \underset{i \in D}{max} \text{客戶中增加量(or 比例)最大之} \overset{\text{Travel time}}{\text{“費用”}}
\]
\begin{itemize}
  \item performance metric 亦傾向縮短總路經長
  \item 此 program 解出之路徑可與“min總路徑長”之結果比較
  \item 費用之分攤則依前頁例子中所述之法(可視每一 $U_i$ 至 $u_i$ 間有一單位流量需求,各 link $l$ 上各乘客之分攤費用為 $\frac{1}{g_l}$ (link $l$上之流量)
  \item 基本model不可慮各客戶之 pick up 及 ETA 條件限制,但須考慮前述 $U_i$ 先於 $u_i$ 之限制
  \item 未來可考慮 $max min$ 各客戶所減少支付(因共乘分攤)車資之比例
  \item 未來亦可結合 admission control,各客戶提出“增額上限”及“節約下限”之條件
\end{itemize}

\subsection*{關鍵字}
\begin{enumerate}
  \def\labelenumi{\arabic{enumi}.}
  \item 全程最短路徑(必訪中繼短之最短路徑問題為 NP-Complete)
  \item 各別 QoS;乘客間可配性(分數)
  \item 各別犧牲程度
  \item 分組機制
  \item 3與4之公平性
  \item M to M
  \item MP方法;先接後送之限制
  \item 最佳 pick up 點之計算
  \item 提前最大化
  \item 送餐服務(MP, LR)
  \item U-bike (MDP)
  \item 隨時重算(以當下情形最為初始狀態)
\end{enumerate}
\end{document}