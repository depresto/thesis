% !TeX root = ../main.tex

\begin{abstract}

共享經濟是隨著網路以及行動裝置普及下逐漸興起的熱潮,傳統的計程車與共乘服務在共享經濟的風潮下產生出新的商業模式,不同於以往使用者必須到目的地才能知道價錢,或是只能針對特定路線如通勤、通學等才容易有共乘機會;透過線上叫車服務,使用者除了可以事先知道價錢,還可以透過共乘配對服務,自動配對有相近路線的乘客,共同分攤車資,提供使用者更便宜經濟的選擇。

目前的線上共乘配對服務,為共乘平台針對目前行徑中或是配對中的乘客透過運算,找尋最適合的路線與乘客。自動配對的共乘服務中,往往需要多繞路以同時滿足不同乘客間的載運需求,如何讓使用者之間繞路多寡符合公平性,便是重要的挑戰。本研究考慮共享經濟中共乘服務的多乘客路線規劃問題,將乘客可接受的抵達時間、繞多少路的可接受程度,車輛人數限制,以及載客的優先順序納入考量,以最小化最大乘客共乘後節費比例,透過拉格朗日鬆弛法以取得最佳解。

\end{abstract}

\begin{abstract*}

\end{abstract*}