% !TeX root = ../main.tex

\begin{abstract}

共享經濟是隨著網路以及行動裝置普及下逐漸興起的熱潮,傳統的計程車與共乘服務在共享經濟的風潮下產生出新的商業模式,不同於以往使用者必須到目的地才能知道價錢,或是只能針對特定路線如通勤、通學等才容易有共乘機會;透過線上叫車服務,使用者除了可以事先知道價錢,還可以透過共乘配對服務,自動配對有相近路線的乘客,共同分攤車資,提供使用者更便宜經濟的選擇。

在自動配對的共乘服務中,往往需要多繞路以同時滿足不同乘客間的載運需求,如何讓使用者之間的車資分配符合公平性,便是重要的挑戰。本研究考慮共享經濟中共乘服務的多乘客路線規劃問題,將乘客可接受的抵達時間、繞多少路的可接受程度,車輛人數限制,以及載客的優先順序納入考量,以最大化最小的乘客共乘節費比例,並透過拉格朗日鬆弛法以取得最佳解。

\end{abstract}

\begin{abstract*}
  The sharing economy is a boom with the rise of the Internet and mobile devices. Traditional taxis and ride-sharing services have produced new business models under the sharing economy. Unlike in the past, users can know the price only if they go to the destination. It is not easy to have shared ride opportunities if it is not the commutation route for work and school. Instead, in the online ride-hailing service, users can know the price in advance. The service can also automatically match passengers with similar routes in real-time. Sharing the fare provides users with a cheaper and more economical choice.
  
  In the ride-sharing service with automatic matching, detours often happen when there are multiple riders with different needs at the same time. How to make it fair to charge the riders is a significant challenge. This research considers the problem of multi-riders route planning for ride-sharing services. Considering the acceptable arrival time, the acceptable degree of detours, vehicles' capacity, and passengers' priority order constraints. Maximize the minimal rider's savings ratio compared to self ride-hailing with carpooling. Use the Lagrangian relaxation method to obtain the best solution.
\end{abstract*}