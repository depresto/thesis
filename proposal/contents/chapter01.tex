% !TeX root = ../main.tex

\chapter{Introduction}

\section{Background}

近年來,隨著網際網路的普及以及幾乎人手一支的行動裝置帶動之下,資訊分享無遠弗屆,開始有人將閒置資源放到網路上與人們共享與交換,然而與陌生人共享意味者風險,隨著使用者評比系統的出現,共享平台開始可以透過評比知道使用者好壞,大大降低了網路共享行為的風險\cite{schor_debating_2014},此後這樣的共享平台逐漸推廣到生活的各個層面,蔚為風潮,從食物外送、住宿、乘車都可以見到他的身影,而人們稱呼這樣的新模式為「共享經濟」。

如何定義共享經濟目前並沒有統一的說法,一般來說,共享經濟中的活動主要分為四大類型:商品的再循環、增加耐久財的使用率、服務交換與經營性資產的共享\cite{schor_debating_2014}。這些活動本質上圍繞著「閒置資產」的重新分配\cite{frenken_smarter_2015},不論是透過資源的交換,或從中賺取報酬,只要透過共享經濟平台的媒合,低廉的交易成本使得閒置資產擁有者將資產發揮更大的價值,而需求者能以較低廉的成本運用該資產,如此一來便能增進整體社會資產的使用率,甚至可以「以租代買」,改變以往必須擁有資產才能使用的思維。

共享經濟的閒置資產重新分配精神,為行之有年的計程車與共乘服務注入了活水,相對於傳統的大眾運輸服務,共享經濟下的新型態的運輸服務有以下的特色:
  
1. 有效率的資源使用:傳統的計程車以往需要透過路邊攔車、電話叫車或是計程車隊的無線電派車模式,以台灣為例,根據交通部數據,巡迴攬客、招呼站等候以及定點排班\cite{noauthor_100nianjichengcheyingyunzhuangkuangdiaochabaogao_2012}便佔了超過一半的比例,這樣的叫車模式,需要計程車在路上徘徊,或是在定點等候,容易增加計程車的時間空車率,造成資源浪費;透過叫車平台的派車服務,可以隨時媒合平台上的計程車,使用者也不須在路邊等候,只需要打開手機,便可明確知道計程車抵達時間,旅程時間,減少不確定性同時也可以提升計程車使用率。
2. 動態的服務資源:叫車平台上的司機可彈性決定自己的工作時間,隨時可以透過登入與登出叫車App,決定是否開放載客,因此相對於傳統無線電計程車有固定的派車資源,叫車平台的供給方有非常高度的動態性。
3. 動態共乘:有些叫車App也提供共乘配對服務,如UberPool與Lyft Line,只要開啟App中的共乘選項,便能及時配對有相近路線需求的乘客,不但可以一起分攤車資,還可以減少旅程中空位的浪費,然而對於平台設計者而言,如何在動態的路線需求中,規劃出能滿足各乘客的行車路線,並在媒合各乘客路線的同時,能有合理範圍的繞路。

從以上特性中,我們可以了解到,在新型態的叫車平台上,不論供需雙方都具有高度的動態性與不確定性,在叫車平台上的共乘服務,還需考量隨時從各地出現的路線需求中,找到合適的司機規劃出適合的路線,這對平台方來說,會是一大考驗。

\section{Motivation}

In the early, ride-sharing was necessary to go carpooling by matching at a taxi station, a fixed commuter route or self-organized among people who knew each other \cite{chan_ridesharing_2012}. With the maturity of mobile devices, they are widely used in every corner of life, such as food, clothing, housing, and transportation. Carpooling is a classic example. In the sharing economy era, when one takes out his/her mobile phone, the ride-sharing service can make better use of the "idle assets" such as seats and vehicles. Drivers can now earn more by detour s for ride-sharing. Riders can share the fare with others at a cost-effective price.

However, a trip may consist of many riders with different origins and destinations; charging riders with fairness is a big problem. There are detours and waiting times among the riders, which may be varied. Some people likely detour a lot in a shared ride, and some people hardly detour but have to pay a similar fare. It is very unfair for the riders in the same carpooling. Therefore, making it fair between passengers in the ride-sharing problem is the object of this article.

\section{Objective}

This research aims to maximize the minimal percentage of cost-saving between a rider choose to hail a ride on his/her own and go carpooling. In order to ensure there is no significant deviation from what a rider expects. Constraints are applied with a ride request of origin and destination, car capacity, and detour limitation.

\section{Thesis Organization}

The rest of the paper is organized as follows. We will go through the related work of the carpooling problem in Chapter 2. Chapter 3 will describe the carpooling problem in detail and formulate it into a mathematical model.