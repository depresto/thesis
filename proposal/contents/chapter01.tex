% !TeX root = ../main.tex

\chapter{Introduction}

\section{Background}

In recent years, with the rising of the Internet and mobile devices, information sharing has become a daily thing. Some people began to put idle resources on the Internet to share and exchange with people. However, sharing with strangers means User risk. With the invention of user rating systems, sharing platforms can begin to know whether users are good or bad through the rating, which reduces the risk in online sharing\cite{schor_debating_2014}. As a trend, from food delivery, accommodation to transportation, the sharing platforms seem to be a part of daily life. People call this new concept a "sharing economy."

There is currently no clear definition of the sharing economy so far. Generally speaking, activities in the sharing economy contain four categories: recycling of goods, increasing the use rate of durable goods, exchange of services, and sharing of operating assets\cite{ schor_debating_2014}. All of the activities are about the re-allocation of "idle assets" \cite{frenken_smarter_2015}. No matter the exchange of resources or earning from the sharing platforms, low transaction costs make the idle assets more valuable. The demander can use the asset at a lower cost, increase the overall utilization rate of social assets. "Renting rather than buying" changes the concept of assets.

The spirit of re-allocation of idle assets in the sharing economy has extended the taxis and ride-sharing services. Compared with traditional public transportation services, the new type of transportation services under the sharing economy have the following features:
  
1. Efficient use of resources: Hailing by the roadside, phone appointments are the traditional taxis method. Take Taiwan as an example. According to data from the Ministry of Transport, street hailing, taxi stands, and scheduling \cite{noauthor_100nianjichengcheyingyunzhuangkuangdiaochabaogao_2012} are more than 50\%, which can increase the time vacancy rate of taxis and cause waste of resource. Through the ride-hailing platform, a ride request can be matched on the platform at any time. Riders do not need to wait on the roadside. They only need to use their mobile phones to know the arrival time and journey time, reducing uncertainty and improving car usage rate.
2. Dynamic resources: Drivers on the ride-hailing service can decide their working hours with flexibility by log in and logout. Therefore, compared with traditional taxis with fixed resources, the ride-hailing platform's resources are highly dynamic.
3. Dynamic ride-sharing: Some ride-hailing apps also provide ride-sharing services, such as UberPool and Lyft Line. As long as drivers turned on the ride-sharing option, riders with similar routes can be matched in time. Not only can the fare be shared, but it also can reduce the waste of vacant seats in the journey.

From the above features, we can understand that both the supplier and the demander have a high degree of dynamics and uncertainty on the new type of ride-sharing service. Finding a suitable route from many dynamic constraints will be a big challenge.

\section{Motivation}

In the early, ride-sharing was necessary to go carpooling by matching at a taxi station, a fixed commuter route or self-organized among people who knew each other \cite{chan_ridesharing_2012}. With the maturity of mobile devices, they are widely used in every corner of life, such as food, clothing, housing, and transportation. Carpooling is a classic example. In the sharing economy era, when one takes out his/her mobile phone, the ride-sharing service can make better use of the "idle assets" such as seats and vehicles. Drivers can now earn more by detour s for ride-sharing. Riders can share the fare with others at a cost-effective price.

However, a trip may consist of many riders with different origins and destinations; charging riders with fairness is a big problem. There are detours and waiting times among the riders, which may be varied. Some people likely detour a lot in a shared ride, and some people hardly detour but have to pay a similar fare. It is very unfair for the riders in the same carpooling. Therefore, making it fair between passengers in the ride-sharing problem is the object of this article.

\section{Objective}

This research aims to maximize the minimal percentage of cost-saving between a rider choose to hail a ride on his/her own and go carpooling. In order to ensure there is no significant deviation from what a rider expects. Constraints are applied with a ride request of origin and destination, car capacity, and detour limitation.

\section{Thesis Organization}

The rest of the paper is organized as follows. We will go through the related work of the carpooling problem in Chapter 2. Chapter 3 will describe the carpooling problem in detail and formulate it into a mathematical model.